\documentclass[conference,compsoc]{IEEEtran}

% *** CITATION PACKAGES ***
%
\usepackage[numbers]{natbib}
\ifCLASSOPTIONcompsoc
  % IEEE Computer Society needs nocompress option
  % requires cite.sty v4.0 or later (November 2003)
  \usepackage[nocompress]{cite}
\else
  % normal IEEE
  \usepackage{cite}
\fi


% correct bad hyphenation here
\hyphenation{op-tical net-works semi-conduc-tor}
\usepackage{hyperref}
\hypersetup{
    colorlinks,
    linkcolor={black},
    citecolor={black},
    urlcolor={blue!80!black}
}
\usepackage{footnotebackref}
\usepackage{url}
\usepackage[dvipsnames,table,xcdraw]{xcolor}
\newcommand{\seo}[1]{\textcolor{red}{SP:~~#1}}

\begin{document}

\title{An Analysis of the Correlation Between Writing Styles and Emotional States Using HMMs}

\author{
\IEEEauthorblockN{Mansi Saxena}
\IEEEauthorblockA{msaxena4@ncsu.edu}
\and
\IEEEauthorblockN{Seoyeong Park}
\IEEEauthorblockA{spark43@ncsu.edu}}


\maketitle

% Main idea
% Reddit data from each thread, like /sadness, etc
% Clustering for each thread → important features for each emotion
% MentalBERT, MentalRoberta
% How will we build an HMM? 
% Find relevant papers to this and to different writing styles
% We need different writing styles to differentiate between mental states/ emotions
% Evaluation of MentalRoberta and the HMM
% Use the HMM to build the empathetic text suggestion model. 
% Find more relevant papers




\section{Problem Description}
In today's digital age, online communities like Reddit play a significant role in providing emotional support and a sense of belonging to individuals dealing with various mental health issues. Users often express their feelings, emotions, and experiences on different subreddits, such as r/depression, r/loneliness and r/sadness. Harnessing the valuable data generated within these communities is a crucial step in creating more effective mental health support systems. Such systems can provide users with empathetic responses and resources tailored to their emotional needs, ultimately contributing to a more empathetic and supportive online environment for those struggling with mental health issues.

Our proposal centrally focuses on exploring the relationship between an individual's mental state and their writing style on social media platforms, specifically on a conversational platform like Reddit. It aims to discover whether emotional shifts manifest as discernible changes in the tone, usage of vocabulary, content, patterns, or style of their online posts under the assumption that emotional fluctuations and writing style changes have a connection in online communication. 

The primary challenge is to develop a comprehensive solution that not only analyses the rich textual data from these subreddits but also understands and categorises the diverse range of emotions and mental states expressed by users. To tackle this challenge, we aim to construct a Conditional Random Field (CRF) that captures transitions in writing styles and maps it to transitions in emotional states. This approach challenges the notion that writing styles can be neatly categorised into a binary classification, instead suggesting that they may reflect nuanced emotional states, reflecting subtleties in linguistic expressions. 

Ultimately, we envision using this CRF to create empathetic AI suggestion models that can complement platforms like Reddit. These models aim to provide comforting and supportive responses to users grappling with mental health disorders within these online communities, with the ultimate goal of improving their well-being.

\subsection{Example scenario}
Imagine a Reddit user, Sarah, frequently posting on r/depression. Initially, her posts are filled with despair and use words like "hopeless" and "empty." Over time, her writing style subtly changes. She starts using more positive language, discussing "progress" and "hope." The proposed solution would track these shifts in her writing style using a hidden Markov model. It would correlate these changes with shifts in her mental state, suggesting that Sarah's evolving writing style reflects her emotional improvement. This analysis could help identify users who are transitioning to a more positive mental state, potentially offering them tailored support or resources.

\section{Relevant Literature}
Though there is related work on detecting mental health disorders from social media, there are none that use an HMM to map changes in writing styles to changes in emotional states. \citet{Park+18:mental-health-reddit}'s work contrasts the character of online discourse within mental health communities, namely, "Anxiety", "Depression", and "PTSD", discerns prevalent themes and underscores distinctions in participation and conversational approaches. We find their clustering approach and qualitative study relevant to our problem. An HMM model capable of categorizing users as either depressed or not, much similar to our proposed model, is presented in \citep{Ansari+18:dcr-hmm}. They train the HMM on a dataset where users are presented with content, and their responses are recorded. Though this was published in 2018, we did not find other recent, relevant papers. We are in the process of obtaining the dataset and code for these papers. Ultimately, we aim to build an AI model to enhance empathy in online conversational platforms, as demonstrated by \citet{Sharma+23:human-ai-empathic-conversation}

\section{Problem importance}
This problem is crucial as it addresses the well-being of individuals in online mental health communities, such as Reddit. Our application aims to provide empathetic AI support tailored to users' emotional states, reducing isolation and improving well-being. Its special feature is the analysis of writing style shifts to detect nuanced changes in emotional states. This solution streamlines the process of seeking emotional support, enabling automated, scalable mental health assistance. It can also aid in the early detection of mental health issues, destigmatizing discussions, and exploring mental health trends in online communities through data analysis of Reddit posts.

\section{Method}
We first scrape data from online Reddit communities like r/anger, r/fear, r/sadness and so on. Each of these communities will be an emotional state. We will refer to the work of \citet{Plutchik+Kellerman-13:emotion} to decide the list of emotional states to be included in our model. 

We then perform clustering for posts from each of these communities, similar to \citet{Park+18:mental-health-reddit}, to get distinct and dominant writing styles for each of the communities. We will analyse common words, hashtags, and relevant factors to predict unique writing styles for each mental state. We then employ pre-trained language models like MentalBERT and MentalRoBERTa \citep{Ji+21:MentalBERT} and fine-tune our dataset to predict the emotional state of a user through their online Reddit post. We will use labels from the names of the communities. 

Once we have a distinct writing style for each of the emotional states and a model to classify emotional states, we begin building the dataset for our HMM model. Our original dataset must contain a list of users (without sensitive personal information) and their multiple posts in a community over a time period. For each post, we get its writing style from our clustering model and the emotional state classification from our fine-tuned model. We use these results to create a new dataset of these Reddit posts with a temporal stamp, a clustered writing style and a classified emotion. This dataset is used to train the HMM model, which learns the transitions in writing styles over time and maps them to corresponding changes in emotional states, as is done by \citet{Ansari+18:dcr-hmm}. In this way, we build the HMM. 

We intend to use the HMM to build an AI model that assists users in enhancing empathy in their responses on Reddit. However, the implementation of this is outside the scope of this semester's project. 

\section{Justification}
Alternative methods for emotion detection include sentiment analysis, topic modelling, and the use of sequential models. Sentiment analysis classifies posts based on the emotions expressed in the text, providing a broad understanding of emotional content. However, it often lacks capturing subtle shifts. Topic modelling identifies topics within the text but may not capture subtle nuances in the context associated with mental states. Sequential models may offer better performance; however, these models may require substantial amounts of data and computing resources.

Our idea introduces a novel approach by adopting HMM to map transitions between writing style changes and emotional states. To the best of our knowledge, there is a lack of attempts with this approach in the context of mental state prediction. We believe that HMM can better capture subtle changes in writing style associated with different emotional states, potentially improving accuracy and reliability compared to existing methods. 

As demonstrated by \citet{Sharma+23:human-ai-empathic-conversation}, existing systems often provide corrections to human responses for empathy but may not take into consideration the emotional state of a user, thus unable to adapt well to specific situations. Our approach goes beyond correction and aims at providing tailored mental health support, which is a critical and sensitive domain where empathy is paramount. By integrating HMM-derived insights into the model, our system gains a deeper understanding of users' emotional states, enabling it to offer more contextually relevant and effective responses.

\section{Evaluation}
We will evaluate our model by comparing it with other pre-trained models like BERT. Performance measures would vary; F1 score, accuracy, and AUC-ROC may be considered to see how well the model works in emotional state recognition and the quality of generated empathetic responses. We will also perform qualitative analysis to evaluate our CRF, such as examining the automaton's ability to capture subtle writing style changes to provide a holistic evaluation.

\section{Future work}
The next step would be implementing an AI-text generator that suggests a revision of user text to be more empathetic based on CRF and the results we find in this study. 

\bibliographystyle{IEEEtranN}
\bibliography{Seoyeong}




\end{document}


